\section{Conclusiones}

El modelado de problemas utilizando sistemas de ecuaciones lineales de la forma $Ax = b$ es sumamente útil, principalmente debido a que se presentan de forma frecuente y sus metodologías de resolución han sido ampliamente estudiadas en la literatura. Para resolver este tipo de problemas, en este trabajo solo utilizamos la eliminación gausiana y la factorización LU, aunque cabe resaltar que existen muchos mas métodos que toman provecho de la estructura del problema en cuestión y logran una complejidad temporal y espacial superior. Para problemas 'chicos', estos factores pueden no ser muy importantes, aunque en la practica muchas veces se presentan sistemas donde si se tornan relevantes.

En el desarollo de este trabajo practico utilizamos dos métodos de resolución clásicos, la eliminación gaussiana y la factorización LU. Estos métodos son conocidos por tener una complejidad en \order{n^3}. Sin embargo, al cambiar el vector del sistema lineal $b$, la eliminación gaussiana pierde en su proceso algorítmico información sumamente relevante que podría ser utilizada para evitar las operaciones entre filas utilizadas para encontrar un sistema triangular superior. Aquí es donde entra la factorización LU, que guarda la información de las operaciones entre filas y logra resolver instancias adicionales en \order{n^2}. Este resultado teórico fue luego confirmado experimentalmente.

La eliminación gaussiana y la factorizacion LU tienen una característica en particular. \textbf{MAURO HABLA UN POCO DE PIVOTEO Y TUS CONCLUSIONES SOBRE NUESTRO SISTEMA EN PARTICULAR. TAMBIEN PROPONE COMO MEJORAR LA COMPLEJIDAD ESPACIAL UTILIZANDO EL HECHO DE QUE ES BANDA.}

Para resolver el problema del horno, tuvimos en primer lugar que buscar alguna manera de buscar una isoterma dentro de una aproximación numérica de las temperaturas. Esto en principio es problemático, dado que no siempre las aproximaciones numéricas son buenas, y ademas porque es necesario definir algún tipo de criterio de interpolación al no poder encontrar temperaturas exactamente iguales a las que estamos buscando. Por esta razón definimos dos métodos y luego llevamos a cabo experimentaciones numéricas para definir cual era mejor. 

Nos sorprendió notar que aumentar la granularidad no necesariamente aumenta la calidad de las isotermas encontradas. Al aumentar la cantidad de radios, esto solo sucede cuando aumentamos la granularidad en múltiplos de 2, dado que bajo estas condiciones todos los puntos anteriores al aumento se encuentran contenidos en la nueva discretización.

Por otro lado, muchas veces nos encontramos ante problemas simétricos, donde la temperatura externa es uniforme para todos los ángulos. Aquí es donde notamos que podemos explotar esta simetría y solo considerar un angulo para el sistema. Utilizando esto podemos lograr una precisión mucho mayor de las isotermas para una dada dimension de la matriz A.

Un dato que puede sonar bastante intuitivo, la propagación del calor no es uniforme en una estructura circular. Uno de nuestros algoritmos de búsqueda de isotermas se basaba incorrectamente en el supuesto de que la propagación si era uniforme, razón por la cual el algoritmo de búsqueda 'naive' termino dando mejores resultados.