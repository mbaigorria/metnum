\section{Conclusiones}

El modelado de problemas utilizando sistemas de ecuaciones lineales de la forma $Ax = b$ es sumamente útil, principalmente debido a que se presentan de forma frecuente y sus metodologías de resolución han sido ampliamente estudiadas en la literatura. Para resolver este tipo de problemas, en este trabajo sólo utilizamos la eliminación gaussiana y la factorización LU, aunque cabe resaltar que existen muchos mas métodos que toman provecho de la estructura del problema en cuestión y logran una complejidad temporal y espacial superior. Para problemas 'chicos', estos factores pueden no ser muy importantes, aunque en la práctica muchas veces se presentan sistemas donde si se tornan relevantes.

En el desarollo de este trabajo práctico utilizamos dos métodos de resolución clásicos, la eliminación gaussiana y la factorización LU. Estos métodos son conocidos por tener una complejidad en \order{n^3}. Sin embargo, al cambiar el vector del sistema lineal $b$, la eliminación gaussiana pierde en su proceso algorítmico, información sumamente relevante que podría ser utilizada para evitar las operaciones entre filas utilizadas para encontrar un sistema triangular superior. Aquí es donde entra la factorización LU, que guarda la información de las operaciones entre filas y logra resolver instancias adicionales en \order{n^2}. Este resultado teórico fue luego confirmado experimentalmente. No obstante, a priori, no toda matriz cuadrada admite esta factorización; esto nos ha llevado a estudiar un poco mas la naturaleza del sistema. En concecuencia, tras develar algunas propiedades, hemos podido demostrar que la aplicación de la eliminación gaussiana reduce nuestra matriz a una triangular superior con elementos distintos de cero en su diagonal (de lo que se deduce que es no singunlar) y sin necesidad de aplicar pivoteo total ni parcial. Como hemos visto en las clases teóricas, dichas hipotesis nos llevan a concluir que el sistema siempre tendrá una única solución, y en particular que posee factorización LU.

Otra carácteristica que pudimos comprobar en la matriz del sistema es su condición de \textit{banda-n}, esto a priori, puede no representarnos ventajas para los dos metodos empleados; no obstante somos concientes que usando otras estructuras de datos, y aprovechando la redundancia de ceros acumulados en la esquina superior izquierda e inferior derecha, se podría reducir para casos lo suficientemente grandes, la complejidad espacial (almacenando sólo aquellos elementos que se encuentre dentro del rango de la banda) y también la complejidad temporal, pudiendo ignorar a aquellas filas en las cuales el elemento de la misma columna del pivot actual se encuentra fuera del rango de la banda. 

Para resolver el problema del horno, tuvimos en primer lugar que buscar alguna manera de buscar una isoterma dentro de una aproximación numérica de las temperaturas. Esto en principio es problemático, dado que no siempre las aproximaciones numéricas son buenas, y ademas porque es necesario definir algún tipo de criterio de interpolación al no poder encontrar temperaturas exactamente iguales a las que estamos buscando. Por esta razón definimos dos métodos y luego llevamos a cabo experimentaciones numéricas para definir cual era mejor. 

Nos sorprendió notar que aumentar la granularidad no necesariamente aumenta la calidad de las isotermas encontradas. Al aumentar la cantidad de radios, esto solo sucede cuando aumentamos la granularidad en múltiplos de 2, dado que bajo estas condiciones todos los puntos anteriores al aumento se encuentran contenidos en la nueva discretización.

Por otro lado, muchas veces nos encontramos ante problemas simétricos, donde la temperatura externa es uniforme para todos los ángulos. Aquí es donde notamos que podemos explotar esta simetría y solo considerar un angulo para el sistema. Utilizando esto podemos lograr una presición mucho mayor de las isotermas para una dada dimension de la matriz A.