\documentclass[10pt,a4paper]{article}
\usepackage[paper=a4paper, hmargin=1.5cm, bottom=1.5cm, top=3cm]{geometry}

\usepackage[utf8x]{inputenc}
\usepackage[spanish]{babel}

\usepackage{mathtools}
\usepackage{amsmath}
\usepackage{amsfonts}
\usepackage{amssymb}

\usepackage{xcolor}
\usepackage{listingsutf8}
\usepackage{booktabs}
\usepackage{hyperref}
\usepackage{multirow}

\usepackage{caption}
\usepackage{subcaption}

\usepackage{algorithm}
\usepackage[noend]{algpseudocode}

\usepackage{graphicx}
\usepackage{tikz}
\usepackage{relsize}

\usepackage{chessboard}
\storechessboardstyle{6x6}{maxfield=h8}

\DeclarePairedDelimiter{\ceil}{\lceil}{\rceil}

%\let\NombreFuncion=\textsc
%\let\TipoVariable=\texttt

%\newcommand{\TipoFuncion}[3]{%
  %\NombreFuncion{#1}(#2) \ifx#3\empty\else $\to$ \res\,: \TipoVariable{#3}\fi%
%}

% set the default code style
\lstset{
    frame=tb, % draw a frame at the top and bottom of the code block
    tabsize=4, % tab space width
    showstringspaces=false, % don't mark spaces in strings
    numbers=left, % display line numbers on the left
    commentstyle=\color{green}, % comment color
    keywordstyle=\color{blue}, % keyword color
    stringstyle=\color{red} % string color
}

% mathy stuff
\newtheorem{theorem}{Theorem}[section]
\newtheorem{lemma}[theorem]{Lemma}
\newtheorem{proposition}[theorem]{Proposición}
\newtheorem{corollary}[theorem]{Corollary}

\newenvironment{proof}[1][Demostración]{\begin{trivlist}
\item[\hskip \labelsep {\bfseries #1}]}{\end{trivlist}}
\newenvironment{definition}[1][Definición]{\begin{trivlist}
\item[\hskip \labelsep {\bfseries #1}]}{\end{trivlist}}
\newenvironment{example}[1][Example]{\begin{trivlist}
\item[\hskip \labelsep {\bfseries #1}]}{\end{trivlist}}
\newenvironment{remark}[1][Remark]{\begin{trivlist}
\item[\hskip \labelsep {\bfseries #1}]}{\end{trivlist}}

\newcommand{\qed}{\nobreak \ifvmode \relax \else
      \ifdim\lastskip<1.5em \hskip-\lastskip
      \hskip1.5em plus0em minus0.5em \fi \nobreak
      \vrule height0.75em width0.5em depth0.25em\fi}

\title{Métodos Numéricos \\ TP1}

\newcommand{\order}[1]{$\mathcal{O}(#1)$}

\begin{document}

%% cover page

\maketitle

\bigskip

\begin{table}[h]
\centering
\begin{tabular}{|l l l|}
\hline
Integrante       & \multicolumn{1}{c}{LU}     & Correo electrónico        \\ \hline
Martin Baigorria & \multicolumn{1}{c}{575/14} & martinbaigorria@gmail.com \\ 
Federico Beuter & 827/13                      & federicobeuter@gmail.com \\
Rodrigo Kapobel & 864/13                      & jangamesdev@gmail.com \\ 
Mauro Cherubini & 835/13                      & cheru.mf@gmail.com \\ \hline
\end{tabular}
\end{table}

\begin{center}
\textbf{Reservado para la cátedra}
\end{center}
\begin{table}[h]
\centering
\begin{tabular}{|l|l|l|}
\hline
Instancia       & Docente & Nota \\ \hline
Primera entrega &         &      \\ \hline
Segunda entrega &         &      \\ \hline
\end{tabular}
\end{table}

\vfill
\textbf{Resumen:}
El siguiente trabajo practico tiene como objetivo implementar, utilizar y evaluar el método de eliminación gausiana y la factorización LU para resolver un problema que involucra la propagación del calor en la pared de un horno descripta con un laplaciano. Para ello se discretizara esta ecuación diferencial y luego se planteara un sistema de la forma $Ax = b$ para calcular la temperatura en los diferentes puntos de la discretizacion en la pared del horno.
Se analizaran diferentes escenarios en las condiciones del problema para evaluar en que escenarios una factorización supera a la otra. Se evaluaran algunas cuestiones relacionadas con el problema en sí, como por ejemplo la presición de un algoritmo de búsqueda de isotermas en la pared del horno y la velocidad de convergencia a la isoterma teórica.
Finalmente concluimos que la factorización LU supera ampliamente a la eliminación gausiana en cuanto a complejidad temporal en escenarios donde cambia el vector $b$ de forma recurrente. A su vez, notamos que la presición del algoritmo de búsqueda de isotermas no es estrictamente decreciente en función de la discretización, aunque por supuesto mejora a medida que aumenta en múltiplos de 2.

\textbf{Keywords:} Gaussian Elimination, LU Factorization.

\newpage
\tableofcontents
\newpage

% end cover page

\section{Introduccion}

El 25 de Mayo de 2015 el diario The New York Times publico un articulo titulado "Years Later for Guillermo Vilas, He’s Still Not the One", donde se repasa el rendimiento del tenista argentino durante los anios 1975/1976 y se discute el calculo del ranking de la ATP en ese momento. Aunque hoy en día Vilas es un icono del tenis argentino, nunca logro estar en la cima del ranking de la ATP.

\begin{figure}[H]
  \centering
  \includegraphics[scale=6]{images/nyt}
  \caption{Guillermo Vilas after winning a tournament in Stockholm in 1975. A journalist has asserted that Vilas deserved to be ranked No. 1 during that year. }
\end{figure}

En 2016, un grupo de investigadores argentinos decidió analizar el ranking de la ATP en 1975 y 1976 para determinar si Vilas debio haber sido numero 1. Dado que los rankings no se actualizaban constantemente en ese momento, los investigadores mostraron que de haberse actualizado de forma periódica, Vilas hubiese sido numero 1 por durante 7 semanas en 1975 y 1976.

Existen precedentes donde se actualizo un ranking de tenis de forma retroactiva. Este es el caso de la WTA, que determino que Evonne Goolagong Cawley debió haber sido numero 1 por dos semanas en 1976. Por esta razon el grupo de investigación argentino considera que revisar estos rankings no es un esfuerzo en vano. Cuando buscábamos los datos de la ATP en el 1975/1976, uno de los investigadores de este equipo que contactamos nos comento: \say{Es interesante tu decisión de indagar sobre el tema. Tal vez no estás al tanto del trabajo y lucha que estamos realizando contra la ATP, por el ranking de los 70 en el que perjudicaron a Vilas y muchísimos otros jugadores.}.

En ese momento, el calculo del ranking de la ATP era bastante rudimentario: \say{It was a system based on an average of a player’s results, and it often rewarded top players who played fewer tournaments. Vilas was a workhorse, which is how he managed not to reach
No. 1 in the ATP rankings in 1977, when he won the French Open, the United States Open and 14 other tournaments.} \cite{nyt}.

Los métodos para calcular rankings no solo son relevantes para definir las posiciones de equipos y jugadores en eventos deportivos, sino que aparecen constantemente en todo tipo de situaciones donde se debe imponer algun tipo de orden. Este es el caso por ejemplo de los concursos docentes, donde se ponderan los diferentes antecedentes para decidir cual es el candidato \textit{idoneo} para el puesto.

Otro caso sumamente relevante en cuanto algoritmos de rankeo es el de los motores de búsqueda. Los motores de búsqueda deben encontrar alguna forma de ordenar de forma relevante los sitios web que están relacionados con una consulta. El caso iconico es el de Google con su algoritmo PageRank. Los buscadores antes de 1990 eran sumamente rudimentarios, utilizaban algoritmos de rankeo vulnerables en el sentido que podían ser manipulados y no se explotaba gran parte de la estructura de la web. Esta fue una de las razones por las cuales una consulta no siempre devolvía resultados relevantes. Este fue el caso por ejemplo de algunos buscadores en ese momento como Yahoo! Search o AltaVista.

\begin{figure}[H]
  \centering
  \includegraphics[scale=0.5]{images/altavista}
  \caption{Sitio Web de Altavista, ano 1999.}
\end{figure}

El clásico paper de Brin y Page, \say{The anatomy of large-scale hypertextual Web search engine.} \cite{Brin1998} explica brevemente el origen del motor de búsqueda de Google y del algoritmo PageRank. La idea es básicamente la siguiente, en primer lugar se implementa un crawler distribuido para poder solicitar y armar el grafo de la web. Las palabras de cada sitio son indexadas y guardadas en una base de datos. Al llegar una consulta al buscador, un programa busca la consulta en los indices de paginas. De esta forma llegamos a un conjunto de paginas que están relacionadas con la consulta. Luego, antes de devolverle al usuario los resultados, estas paginas son ordenadas utilizando el famoso algoritmo PageRank. Este algoritmo se basa en la idea de que para medir la relevancia de un sitio se puede usar como proxy la cantidad de sitios que tienen un link al mismo. Para evitar que un usuario malintencionado manipule los resultados del mismo, la relevancia otorgada por un sitio web que linkea a otro es proporcional a su propia relevancia e inversamente proporcional a la cantidad de links (o grado de salida) del mismo.

El presente trabajo practico tendrá como objetivo implementar el algoritmo PageRank para luego utilizarlo para generar rankings de todo tipo, ya sea para ordenar la relevancia de paginas webs o generar rankings deportivos. PageRank es un algoritmo que basa su ranking en encontrar el autovector de una matriz de transiciones. A priori esto puede sonar complicado, pero luego mostraremos que en realidad es bastante simple y elegante. Dado que ordenar la relevancia de millones de sitios web no es un problema trivial, en la practica este problema se resuelve utilizando álgebra lineal y métodos numéricos. Una muy buena introducción teórica se puede encontrar en el trabajo de Bryan y Leise \cite{Bryan2006}. Otros autores como Kamvar, Haveliwala, Manning, Christopher y Golub \cite{Kamvar2003} han buscado otros enfoques y métodos para poder acelerar este algoritmo. La idea es encontrar una forma eficiente de poder computar este modelo, calibrando sus diferentes parámetros de modelado y convergencia para lograr un orden relevante. 

Una vez planteado el procedimiento, experimentaremos con la complejidad temporal de los métodos implementados y evaluaremos los diferentes parámetros a calibrar. Finalmente concluiremos si según el algoritmo PageRank y nuestra matriz de transición Vilas efectivamente debió haber estado en la punta del ATP en 1975/1976. En caso afirmativo, sin dudas nos comunicaremos con la ATP.

\begin{figure}[H]
  \centering
  \includegraphics[scale=0.5]{images/vilasaprueba}
  \caption{Guillermo Vilas apoya este TP.}
\end{figure}
\newpage
\section{PageRank}

\subsection{Modelado para paginas web}

El algoritmo PageRank fue ideado en un principio para buscar de darle alguna medida de relevancia a los sitios web en internet. El mismo tiene dos interpretaciones equivalentes, que serán expuestas a continuación.

El problema se modela a partir de un grafo $G(Web,Links)$ donde $Web$ es el conjunto de sitios web y $Links$ es la cantidad de conexiones entre sitios. Consideremos que toda pagina web $u \in Web$ esta representada por un vértice y la relación entre paginas por un link con una arista. Una representación posible del grafo es mediante matrices de adyacencia. Definimos la matriz de adyacencia o conectividad $W \in \{0,1\}^{n \times n}$ de forma tal que $w_{ij} = 1$ si la pagina $j$ tiene un link a la pagina $i$ y $w_{ij} = 0$ en caso contrario. Por lo tanto, la cantidad de paginas a las que la pagina $u$ apunta ($d_{out}(u)$) se puede calcular como $n_j = \sum_{i=1}^{n} w_{ij}$.

\subsubsection{Propiedades}

Sea $x_j$ el puntaje asignado a la pagina o vértice $j \in Web$ y otra pagina $u \in Web$. La idea es buscar una medida que cumpla con las siguientes propiedades:
\begin{itemize}
  \item La relevancia de todo sitio web es positiva.
  \item La relevancia de un sitio web debe aumentar a medida que mas sitios unicos lo apuntan.
  \item La relevancia derivada de otro sitio web debe depender de su propia relevancia. Es decir, es mas valioso que me linkee un sitio relevante que uno no relevante. En caso de no cumplirse esta propiedad, el ranking seria fácilmente manipulable al permitir que un usuario cree muchos sitios que linkeen a uno para darle relevancia.
  \item La relevancia de todos los sitios web debe sumar uno. De esta manera estamos ante una distribución de probabilidad de los sitios. Mas adelante veremos que al interpretar esto mediante Cadenas de Markov existe una interpretación directa: la relevancia se puede ver como la proporción del tiempo total que un usuario pasa en ese sitio.
\end{itemize} 

Por lo tanto, estamos buscando una medida de relevancia tal que la importancia obtenida por la pagina $u$ obtenida por el link de la pagina $v$ sea proporcional a la relevancia de $v$ e inversamente proporcional al grado de $v$. El aporte del link de $v$ a $u$ entonces es $x_u = x_v / n_v$. Luego, sea $L_k \subseteq Web$ el conjunto de paginas que tienen un link a la pagina $k$. Por lo tanto, la relevancia total de un sitio sera:

\begin{eqnarray}
x_k = \sum_{j \in L_k} \frac{x_j}{n_j},~~~~k = 1,\dots,n. \label{eq:basicmodel}
\end{eqnarray}

Notar que esta es de cierta manera una definición recursiva. La relevancia de un sitio $u$ puede depender de la relevancia de un sitio $v$, y luego la de $v$ puede depender de la de $u$. A priori calcular la relevancia de un sitio puede parecer sumamente complicado, pero luego veremos que al plantearlo como un sistema de ecuaciones esta dificultad per se ya no se presenta.

Definimos entonces una matriz de transición o adyacencia con pesos en las aristas $P \in \mathbb{R}^{n \times n}$ tal que $p_{ij} = 1 / n_j$ si $w_{ij} = 1$ y $p_{ij} = 0$ en caso contrario. Luego, el modelo planteado en (\ref{eq:basicmodel}) para toda pagina web se puede expresar $Px = x$ donde $x \in \mathbb{R}^n$. Notar que esto es equivalente a encontrar el autovector de autovalor 1 tal que $x_i > 0$ y $\sum_{n=1}^{n} x_i = 1$. Notar que si logramos probar que bajo ciertas condiciones nuestra matriz de transición tiene autovalor 1, el signo de todos los elementos de un autovector es el mismo y la dimension del autoespacio es 1 ya tenemos un ranking valido. Esto se debe a que cualquier autovector puede ser reescalado a uno de norma unitaria con $x_i \geq 0$.

\pagebreak

\subsubsection{Existencia y Unicidad}

Bryan y Leise \cite{Bryan2006} analiza y prueba las condiciones bajo las que podemos garantizar que:
\begin{itemize}
\item La matriz de transición tiene autovalor 1.
\item La dimension del autoespacio asociado al autovalor 1 es 1. Es deseable que el ranking asociado a una matriz de transición sea único.
\item El signo de todos los elementos del autovector asociado al autovalor 1 es el mismo.
\end{itemize}

Veamos bajo que condiciones nuestra matriz de transición cumple con estas propiedades:

\begin{definition}
Una matriz cuadrada se llama estocástica por columnas si todos sus elementos son positivos y la suma de cada columna es igual a 1.
\end{definition}

A partir de esta definición se puede probar la siguiente proposición:
\begin{proposition}
Toda matriz estocástica por columnas tiene a 1 como autovalor.
\end{proposition}

Esto significa que si no existen \texttt{dangling nodes}, es decir, vértices con $d_{out} = 0$, podemos garantizar que nuestra matriz de transición es estocástica por columnas.

Notar que bajo las condiciones actuales no podemos garantizar que si existe el autoespacio asociado al autovalor 1, el mismo tenga dimension 1. Intuitivamente, esto se debe a que el grafo de la web puede tener varias componentes conexas. Como comparamos sitios web que no están relacionados? Justamente la relación, ya sea directa o indirecta mediante transitividad me da algún tipo de relación de orden. Al no tener una relación de orden entre dos sitios web bien definida, es razonable que existan múltiples autovectores, es decir, rankings. Esto se puede ver claramente en la pagina 4 del paper de Bryan y Leise \cite{Bryan2006}.

Por lo tanto, la idea es básicamente buscar algún tipo de transformación relevante de mi matriz de transición que me permita garantizar que no voy a tener \texttt{dangling nodes} y ademas que solo tenga una componente conexa, es decir, que el grafo sea conexo. Definimos la siguiente matriz de transición, donde $v \in \mathbb{R}^{n \times n}$, con $v_i = 1 / n$ y $d \in \{0,1\}^n$,  $d_i = 1$ si $n_i = 0$ y $d_i = 0$ como:

\begin{eqnarray*}
D & = & v d^t \\
P_1 & = & P + D.
\end{eqnarray*}

De esta manera, en caso de tener una pagina web que es un \texttt{dangling node}, le asignamos un link uniforme a todos los sitios web $u \in Web$. Una interpretación equivalente es tomar a la matriz de transiciones como la matriz que describe una Cadena de Markov, donde el link pesado representa la probabilidad de dirigirse de una pagina a la otra. Por lo tanto, esta transformación se puede interpretar como que que existe una probabilidad uniforme de ir de uno de estos sitios a cualquiera de la web. Esto normalmente se conoce como el \texttt{navegante aleatorio}.

Tambien podemos considerar la posibilidad de que el navegante aleatorio se dirija a una pagina web que no esta linkeada a la pagina a la que esta actualmente. Este fenómeno se conoce como teletransportación. Para incluirlo al modelo, tomemos un numero $c \in [0,1]$ y transformemos la matriz de transiciones de la siguiente manera, donde $\bar{1} \in \mathbb{R}^n$ es un vector tal que todos sus componentes valen 1:

\begin{eqnarray*}
E & = & v \bar{1}^t \\
P_2 & = & cP_1 + (1-c)E,
\end{eqnarray*}

Notar que en caso de tener $c=1$, estamos en la matriz de transición sin teletransportación. Por otro lado, si $c=1$ estamos en el caso donde solo hay teletransportación y no importa la estructura del grafo de la web.

Esta nueva matriz de transición, dado que es estocástica por columnas y no tiene \texttt{dangling nodes}, nos garantiza que la dimension del autoespacio generado por el autovector de autovalor 1 es unitaria. Solo nos falta mostrar que todo autovector tiene todos sus elementos del mismo signo. Es facil probar la siguiente proposicion:

\begin{proposition}
Si la matriz M es positiva y estocástica por columnas, entonces todo autovector en $V_1(M)$ tiene todos sus elementos positivos o negativos.
\end{proposition}

Por lo tanto, ya probamos la existencia del autovector de norma 1 asociado al autovalor 1 de la matriz de transición transformada. El siguiente lema nos garantiza su unicidad. Su respectiva demostración se encuentra nuevamente en la pagina 7 del paper de Bryan y Leise \cite{Bryan2006}.

\begin{lemma}
\item Si M es positiva y estocástica por columnas, entonces $V_1(M)$ tiene dimension 1.
\end{lemma}

\subsection{Modelado para Tenis}

Hacer referencia al paper de Govan et al. Explicar existencia y unicidad haciendo referencia a las pruebas de modelado de paginas web.

\subsection{Eliminacion Gausiana}

Esta seccion solo la pongo para que la consideres. Se podra hacer eliminacion gausiana con pivoteo para (P-I)x = 0? Igual si es posible es de orden cubico, con la web de millones de paginas se te va al carajo. Es solo para enriquecer la discusion.

\subsection{Representacion del grafo}

Ya hemos demostrado las condiciones necesarias para poder obtener el autovector asociado al autovalor dominante de una matriz de Markov. 
Ahora debemos proceder a calcular el mismo. Para esto, tenemos que tener en cuenta las cualidades del sistema y el método de resolución del algoritmo. Recordemos que en general, el grafo que representa la web tenderá a ser desconexo y muy grande, es decir, que podrán existir dos o mas rankings diferentes. Por lo tanto la matriz 
de transiciones puede ser muy esparsa e inclusive puede suceder que una página no tenga links de salida, dando lugar a dangling nodes. Para solucionar estos inconvenientes, con lo visto anteriormente disponemos de dos soluciones. Para los dangling nodes, la solución consiste en sumar una columna con probabilidad 1/n a la columna de ceros, esto en si, se puede interpretar como la probalidad de navegación aleatoria que previamente describimos. Aunque con esto no solucionamos el problema de la esparsidad de la matriz en si y el de poder tener mas de un ranking diferente. Para esto último, se agregó la matriz de probabilidad de teletransportación.

Dada esta definición, la matriz de transiciones resultante no es esparsa. 
Para sistemas muy grandes, esto puede resultar contraproducente a la hora de obtener el autovector asociado, dado que la complejidad espacial y temporal aumenta  considerablemente con la cantidad de información representada en la matriz. Sin embargo existe un resultado que podremos utilizar para mejorar la eficiencia del algoritmo en términos de complejidad temporal y espacial. El mismo se basa en la idea de Kamvar et al. \cite[Algoritmo 1]{Kamvar2003} para el calculo del autovector. Este resultado nos permite utilizar la matriz original de transiciones sin modificar en lo absoluto, pero si cambiando su representación, valiendonos de una buena estructura para almacenar las entradas de la misma. 

Las cualidades de la matriz hacen que sea razonable intentar pensar en una forma de representar solo las entradas que no sean ceros, y dado que la matriz suele ser esparsa, la misma contendrá muchos ceros que podrían no ser representados. Para esto optamos por una de entre las 3 siguientes estructuras de representación:

\begin{itemize}
\item Dictionary of Keys ($DOK$)

\item Compressed Sparse Row ($CSR$)

\item Compressed Sparse Column ($CSC$)
\end{itemize}

De todas estas representaciones posibles, para este t.p optamos por $CSR$. Aún así no haremos una elección sin una justificación apropiada del porque consideramos que es la mejor para nuestro trabajo, dado que como en toda estructura de datos, siempre existen pros y contras. Nos encargaremos en lo que sigue de exponer estos detalles para dejar en claro nuestro punto de vista. 

\begin{itemize}
\item Dictionary of Keys ($DOK$)
\end{itemize}
Consiste en un diccionario que mapea pares de fila-columa a la entrada. No se representan las entradas nulas. El formato es bueno para gradualmente construir una matriz esparsa en orden aleatorio, pero pobre para iterar sobre valores distintos de cero en orden lexicográfico. Uno construye típicamente una matriz en este formato y luego se convierte en otro formato más eficiente para su procesamiento.

\begin{itemize}
\item Compressed Sparse Row ($CSR$)
\end{itemize}
Pone las entradas no nulas de las filas de la matriz en posiciones de memoria contiguas. Suponiendo que tenemos una matriz dispersa no simétrica, creamos vectores: uno para los números de punto flotante ($val$), y los otros dos para enteros ($col\_ind$, $row\_ptr$). El vector $val$ almacena los valores de los elementos distintos de cero de la matriz, de izquierda a derecha y de arriba hacia abajo. El vector $col\_ind$ almacena los índices de columna de los elementos en el vector $val$. Es decir, si $val(k) = a_ij$ entonces $col\_ind(k) = j$. El vector $row\_ptr$ almacena los lugares en el vector $val$ que comienza y termina una fila, es decir, si $val(k) = a_ij$ entonces $row_ptr(i) \leq k \leq row\_ptr(i+1)$. Por convención, se define $row\_ptr(n+1) = nnz$, en donde $nnz$ es el número de entradas no nulas en la matriz. Los ahorros de almacenamiento de este enfoque es significativo. En lugar de almacenar elementos $n^2$, solamente necesitamos $2nnz + n$ lugares de almacenamiento.

Veamos con un ejemplo como seria la representacion:
\\\\
$\hspace{3.2cm}\begin{pmatrix} 0 & 0 & 0 & 0 \\ 5 & 8 & 0 & 0 \\ 0 & 0 & 3 & 0 \\ 0 & 6 & 0 & 0 \\ \end{pmatrix}$
\\\\
Es una matrix de 4x4 con 4 entradas no nulas. Luego:

   $val$  = [ 5 8 3 6 ]
   $row\_ptr$ = [ 0 0 2 3 4 ]
   $col\_ind$ = [ 0 1 2 1 ]

\begin{itemize}
\item Compressed Sparse Column ($CSC$)
La idea es analoga a $CSR$, pero la compresion se hace por columnas es decir, si $CSR$ comprime $A$, $CSC$ comprime $A^t$  
\end{itemize}

Sobre la matriz definida para $CSR$, con $CSC$ obtenemos lo siguiente

   $val$  = [ 5 8 6 3 ]
   $col\_ptr$ = [ 0 1 3 4 4 ]   
   $row\_ind$ = [ 1 1 2 3 ]

Todos los resultados anteriores permiten evitar representar valores nulos.	
El motivo de nuestra elección se debe a que $CSR$ ofrece una buena representación de las filas de la matriz y es más eficiente a la hora de hacer operaciones del tipo A*x (matriz-vector) que es lo que nos interesa en el método de la potencia que realiza pageRank. $CSC$ en cambio, es efectiva para el producto x*A (vector-matriz) dado que la misma ofrece una mejor representación de las columnas. En contra partida, tanto $CSR$ como $CSC$, no permiten construcción incremental aleatoria, que si ofrece $DOK$, es decir, que cambios a la esparsidad de la matriz son costosos. En general están pensadas para ser estáticas, pero esto no es un inconveniente en nuestro caso, dado que no se realizaran cambios en la esparsidad de la matriz durante el proceso.

En el presente trabajo utilizaremos la idea de Kamvar et al. \cite[Algoritmo 1]{Kamvar2003} para el calculo del autovector valiendonos de nuestra estructura de representación elegida y compararemos los resultados con el algoritmo standard para mostrar que al final de cuentas, si el sistema es muy grande y esparso, puede resultar muy beneficioso en terminos de complejidad espacial y temporal.

\subsection{Computo: Método de la Potencia}

\textbf{LO SIGUIENTE ES PARA MAURO. HACELO EN TE RO (SI ENTERO), NO SEAS PAJERO. QUIERO LA SECCION BIEN COMPLETA CON LOS PUNTOS OPCIONALES, SOLO TENES QUE HACER ESTO. IMAGINATE QUE YO ME MORFE ESCRIBIR TODO.}

La pregunta mas importante del trabajo, dado la matriz de transiciones $P$ que garantiza la existencia y unicidad del autovector de norma 1 asociado al autovalor 1, como lo computamos?

\subsubsection{Correctitud}

La idea básicamente es generar la secuencia $x_k = Px_{k-1}$ y tomar $k\to\infty$. Se puede probar que para este caso no importa el valor inicial que asignemos a la secuencia $x_0$, el vector converge al autovector asociado al mayor autovalor de P. Se puede probar que todo autovalor $\lambda$ de $P$ satisface que $|\lambda| < 1$.

Otra propiedad interesante es que el método de la potencia va a converger de forma asintotica siguiendo $\norm{Px_k - q}_1 \approx |\lambda_2| \norm{x-q}_1$ donde $\lambda_2$ es el segundo autovalor mas grande de P. \textbf{Mauro revisa esto y fijate si sirve y se puede hacer algún criterio de parada copado.}

\subsubsection{Valor Inicial}

Elijo uno uniforme (todos con la misma relevancia y norma 1), o uno bien contra las esquinas (obvio que no)? Con esto hay que experimentar un cachito igual.

\subsubsection{Criterio de Parada}

Diferencias entre normas 1 o hay algo mejor?

\subsubsection{Complejidad}

Cual es la complejidad? Discutir que para deportes mucho no importa, pero para paginas web si.

\subsubsection{Otras propiedades}

Mirar esto antes de resolver lo siguiente...

Sin embargo, en Kamvar et al. se propone una forma alternativa de computar la secuencia. Este resultado debe ser utilizado para mejorar el almacenamiento de los datos. Esta relacionado con el punto de representacion del grafo esto? Ni idea.

... seguir

\textbf{LO SIGUIENTE ES PARA MAURO. HACELO, NO SEAS PAJERO. SI, TE LO DIGO DENUEVO PORQUE SE QUE LO VAS A TRATAR DE EVITAR. NO TE PEDI MUCHO. ESTO ES PARA EL FINAL IGUAL, CUANDO YA TERMINASTE TODO LO DE ARRIBA.}

\begin{itemize}
\item Demostrar que los pasos del Algoritmo 1 propuesto en Kamvar et a. son correctos y computan $P_2x$.
\item Establecer una relacion con la proporción entre $\lambda_1 = 1$  y $|\lambda_2|$ para la convergencia de PageRank.
\end{itemize}
\newpage
\section{Experimentación}

\subsection{PageRank}
\subsubsection{Complejidad}
tiempo de computo en funcion de size del grafo, eje x, cantidad de sitios web, eje y, tiempo en ms a convergencia.

\subsubsection{Casos Patologicos}
Caso particular chiquito, página 3. Fijate el parrafo que arranca en A simple apprroach...... y despues This approach ignores that... La idea es armar el mismo grafo y mostrar el mismo ejemplo jaja

\subsection{páginas Web}

\subsubsection{Comparacion PageRank vs In-Deg (RODRI)}
Comparar solo los rankings, nada de complejidad. Podes mencionar que In-Deg usa un algoritmo \order{n \times log(n)}, pero nada mas. Comparar top 10 con los dos y discutir diferenciias.

\subsubsection{Manipulacion}
página 5, ejercicio 1. La idea es que plantees un caso de un tipo que quiere manipular el ranking, mostra que aunque agregues miles de nuevas páginas apuntando no podes hacer demasiado, hacelo en funcion de la cantidad de páginas que agregas?

Se puede manipular entonces o no? Agarra, en el eje x pone cantidad de sitios web que apuntan solamente al sitio u que le quiero subir el ranking, y en el eje y el ranking de ese sitio. Fijate que aumenta, y fijate si podes hacer algun tipo de curva de nivel con c (cuanto mayor c, mas manipulable es la cosa). Citar el paper de Sergei y Brin, que dicen que hacen promedios de muchas cosas en la practica para evitar este problema. Usan muchos criterios promediados.

\subsection{Ranking ATP}

Empezemos con el apartado que seguro el lector más esperaba de todo el t.p..Vilas fue o no 1ro entre 1975 y 1977? Esta pregunta podemos contestarla. Pero previo a esto, necesitamos poner al lector al tanto de la situación. Empecemos viendo los rankings oficiales de la época.


En 1975, Vilas llegó a la primera final de un Grand Slam, el Roland Garros, en donde fue derrotado por el sueco Björn Borg
y cuartos de final de Wimbledon.


Este es el top 10 para 1975 segun la ATP, donde Guillermo Vilas se hubica 2do:

\begin{eqnarray*}
173 & Jimmy & Connors \\
127 & Guillermo & Vilas \\
34 & Bjorn & Borg \\
19 & Arthur & Ashe \\
225 & Manuel & Orantes \\
210 & Ken & Rosewall \\
144 & Ilie & Nastase \\
180 & John & Alexander \\
318 & Roscoe & Tanner \\
309 & Rod & Laver 
\end{eqnarray*}

En 1976 la ATP hubica a Vilas 6to dentro del top 10:

\begin{eqnarray*}
173 & Jimmy & Connors \\
34 & Bjorn & Borg \\
144 & Ilie & Nastase \\
225 & Manuel & Orantes \\
288 & Raul & Ramirez \\
127 & Guillermo & Vilas \\
1 & Adriano & Panatta \\
134 & Harold & Solomon \\
87 & Eddie & Dibbs \\
41 & Brian & Gottfried 
\end{eqnarray*}

En 1977 Vilas se ubica segundo, por debajo de Connors. Recordemos que esto hasta el día de hoy sigue creando polémicas debido a los muy buenos resultados obtenidos por Vilas en aquel año comparados con los del estadounidense y que luego comentaremos.
Veamos el top 10 oficial para este año:

\begin{eqnarray*}
173 & Jimmy & Connors \\
127 & Guillermo & Vilas \\
34 & Bjorn & Borg \\
371 & Vitas & Gerulaitis \\
41 & Brian & Gottfried \\
87 & Eddie & Dibbs \\
225 & Manuel & Orantes \\
288 & Raul & Ramirez \\
144 & Ilie & Nastase \\
81 & Dick & Stockton 
\end{eqnarray*}

Todos los años tienen como lider indiscutido al estadounidense Jimmy Connors.

\subsubsection{Ranking ATP oficial vs. Ranking PageRank vs. Sort por diferencia de victorias/derrotas}

Introducimos aquí un algoritmo de rankeo con un criterio de ordenamiento basado en victorias/derrotas, que además, si hay empate define por diferencia de puntos, aunque para el caso de tenis y por como esta definido el sistema de puntos del mismo no tiene ninguna relevancia. Solo nos servirá para poder hacer un análisis cualitativo de las virtudes de pagerank sobre algoritmos más básicos y comparar estos resultados con el ranking oficial y poder obtener así alguna conclusión sobre los mótivos principales de la investigación.

Veamos que obtuvimos en cada año con este algoritmo, empezando por 1975:

\begin{eqnarray*}
id & PG & PP \\
19 & 96 & 18 \\
225 & 90  & 20 \\
127 & 86 & 18 \\
173 & 79 & 11 \\
144 & 88 & 22 \\
34 & 82 & 17 \\
288 & 70 & 28 \\
180 & 65 & 23 \\
318 & 64 & 24 \\
155 & 60 & 21 \\
\end{eqnarray*}

Podemos ver que Vilas no aparece segundo, si no tercero. Connors fue desplazado al 4to lugar y en primer lugar aparece Arthur Ashe quien ocupaba el lugar que Connors ocupa ahora. Podemos ver varios cambios relacionados al top 10 oficial. 

Todos ellos particularmente relacionados al hecho de que como se habia anticipado la cantidad de victorias sobre derrotas conformaria otro ranking diferente en el cual no importa exactamente que clase de victorias a conseguido o derrotas sufrido un determinado participante. Podemos ver que Vilas con más victorias sobre derrotas que Connors, se encontraba por debajo de él en el ranking oficial. Esto se debe al sistema de puntos manejados por el ATP donde se suma más puntaje cuanto mas avancemos en un torneo y cuantas más finales ganemos y por la jerarquía de ese torneo. Claramente no alcanza con ganar un solo torneo importante. Debemos tener continuidad y participación.

Observemos el ranking de 1976:

\begin{eqnarray*}
id & PG & PP \\
173 & 86 & 9 \\
127 & 83 & 20 \\
144 & 73 & 15 \\
288 & 91 & 33 \\
225 & 76 & 19 \\
87 & 83 & 28 \\
34 & 57 & 12 \\
318 & 71 & 27 \\
380 & 75 & 32 \\
134 & 70 & 27 \\
\end{eqnarray*}

Vemos que Vilas avanzó del 6to lugar al 2do puesto que era ocupado por  Bjorn Borg y que fue desplazado al 7mo. 

Si analizamos un poco los partidos veremos que Vilas fué derrotado por Bjorn Borg 3 veces (Winbledon, Dallas WCT, Sao Paulo WCT) mientras que Vilas nunca pudo derrotarlo. Contra Connors se enfrentó una vez y fue derrotado (US Open) al igual que Bjorn Borg, que se enfrentó en 3 ocaciones contra Connors y cayó en la misma cantidad...esto nuevamente es un indicador de que determinadas victorias son más importantes que otras.

Veamos que sucede en 1977:

\begin{eqnarray*}
id & PG & PP \\
127 & 126 & 14 \\
41 & 105 & 22 \\
34 & 71 & 8 \\
173 & 69 & 15 \\
87 & 77 & 28 \\
371 & 60 & 15 \\
225 & 65 & 25 \\
134 & 64 & 27 \\
288 & 61 & 24 \\
380 & 66 & 30 \\
\end{eqnarray*}

Ok. Tomemoslo con calma. Vilas aparece primero..pero esto no es indicador absoluto de que la ATP cometió un error debido a la falta de información que provee. Aunque si nos da indicios de lo que realmente pasó.

La diferencia de partidos ganados sobre perdidos con respecto a cualquier otro competidor es bastante significativa. Podemos ver que Connors fué desplazado al 4to lugar. Segundo se ubica Brian Gottfried, que durante ese año tuvo un gran desempeño, entre los que se encuentra su victoria sobre Vilas en la final del Roland Garros. 

Todos estos datos nos permiten dar una idea de hacia donde vamos..las posiciones parecen estar relacionadas con el desempeño del competidor durante ese año, es decir, cuantas finales disputó o que tan lejos avanzó en un torneo y a que rivales venció.

A priori..cantidad de victorias está relacionado con cantidad de puntos..pero esto no es una regla general y depende como mencionamos del sistema de puntajes. Para poder asemejar a la relevancia que la ATP le da a los torneos tenemos que utilizar un algoritmo que aproveche esa carácteristica lo mejor posible. Para esto haremos uso del modelo GeM y analizaremos sus resultados. Esperariamos que los mismos se parezcan al ranking oficial con ciertas variaciones, pero como minimo se mantengan los mismos en el top 10 que el ranking oficial. 

Utilizaremos los siguientes parámetros para generar los 3 rankings:

$c$ = 0.85
$\delta$ = 0.00001

Además, como indicamos en el diseño del sistema, usaremos una matriz de personalización uniforme, dado que no nos interesa que influyan sobre los resultados ninguna información estadistica de un torneo anterior, por el simple hecho de que evaluamos a los jugadores desde cero cada año.

Avancemos sobre los resultados, comenzando con 1975:

\begin{eqnarray*}
id & puntaje \\
19 & 0.033172 \\
34 & 0.030089 \\
225 & 0.026483 \\
144 & 0.026254 \\
127 & 0.023572 \\
173 & 0.021918 \\
180 & 0.017379 \\
318 & 0.015914 \\
309 & 0.011614 \\
210 & 0.011055 \\
\end{eqnarray*}

Vemos que Vilas se ubica en el 4to puesto, Connors fué desplazado al 5to puesto y por arriba se ubican Ashe en primer puesto y Borg en segundo lugar. 

Si analizamos los partidos vemos que Ashe se enfrentó en una sola oportunidad a Connors y logró derrotarlo, en lo que fué la final Wimbledon. Además se enfrento en 7 oportunidades a Borg y logró vencerlo en 4 (Wimbledon, Barcelona WCT, Dallas WCT, Munich WCT), siendo Wimbledon el más importante de los 7 enfrentamientos.

Borg obtuvo la final de Roland Garros frente a Vilas como máximo logro. 

Mientras tanto Manuel Orantes hizo lo propio para obtener el 3er lugar contra Vilas derrotandolo en las 4 oportunidades que se encontraron y derrotando a Connors en su único enfrentamiento. 
No es una tarea fácil determinar todas estas congruencias pero con un simple vistazo a los partidos jugados y los torneos en los que se enfrentaron parece claro que el ranking es elocuente.

Veamos el ranking de 1976:

\begin{eqnarray*}
id & puntaje \\
173 & 0.033300 \\
144 & 0.031773 \\
288 & 0.029343 \\
87 & 0.026328 \\
41 & 0.025002 \\
134 & 0.024189 \\
127 & 0.024006 \\
34 & 0.023806 \\
225 & 0.020587 \\
1 & 0.014804 \\
\end{eqnarray*}

No parace haber mucho que analizar con respecto al ranking oficial. Vilas se encuentra una posición mas abajo. Connors sigue siendo lider indiscutido y la otras posiciones relativas no han cambiado demasiado. Podriamos resaltar el caso particular de Borg que no tuvo malos resultados, destacando su final ganada contra Ilie Nastase en Wimbledon, una final perdida en el US Open contra Connors y Cuartos de final en un Roland Garros. Pero si miramos el mismo ranking pero por diferencia de ganados/perdidos veremos que Borg se encuentra casi en la misma posición debido a la poca cantidad de partidos ganados con respecto a los perdidos. 

Concluyamos con el análisis de 1977: 

\begin{eqnarray*}
id & puntaje \\
127 & 0.037046 \\
41 & 0.035463 \\
34 & 0.029461 \\
173 & 0.027319 \\
134 & 0.024189 \\
87 & 0.023230 \\
81 & 0.021816 \\
371 & 0.017483 \\
288 & 0.016667 \\
225 & 0.016318 \\
\end{eqnarray*}

Por una diferencia significativa, Vilas desplaza del primero puesto a Connors para quedarse el con el trono. Pero hay una razón bastante justificada para que esto suceda.
En 1977 Vilas conquistó 17 torneos (récord todavía vigente), se consagró en Roland Garros frente a Brian Gottfried por 6-0 6-3 6-0 y el US Open frente a Jimmy Connors por 2-6 6-3 7-6 6-0, fue finalista de Australia donde cayó frente a Roscoe Tanner por 3-6 3-6 3-6, logró una seguidilla de 50 partidos consecutivos sin conocer la derrota y, durante esos doce meses, ganó 145 de los 159 encuentros que jugó (91,1\%).
Jimmy Connors, en 1977 no obtuvo ningún torneo de Grand Slam y ganó ocho torneos, menos de la mitad de los logrados por Vilas. 
Parece lógico que Connors incluso este unos escalones más abajo en el ranking. Además el ranking por partidos ganados/perdidos para los primeros 4 es identico. Esto no puede hacer más que confirmar que Vilas fué primero indiscutido en 1977 por una amplia diferencia, sobre todo por el bajo rendimiento de Connors durante ese año. 

Como comentamos al principio, la cantidad de partidos ganados y los rivales derrotados son factores importantes a la hora de generar el ranking con un metodo como GeM. Las condiciones usadas son muy similares a la forma de rankeo que utiliza la ATP, por este motivo podemos concluir que el ranking de la ATP de 1977 con seguridad tiene un error de computo. Si la ATP hubiese cambiado a un sistema como el GeM pero con un sistema de puntos más especializado y las condiciones fueran retroactivas, hoy el número 1 indiscutido de 1977 sería sin lugar a dudas el gran willy.

\subsubsection{Eleccion del factor de 'teletransportacion' c (RODRI)}
probar relevancia a medida que cambias ese valor = 0.85, creo que c.

Citar paper de google, que usan 0.85. Discutir que si c es uno, ignoras la estructura del grafo al hacer el ranking, todos rankean igual.

página 6.... This is the ultimately egalitarian case: the only... blah. La idea es jugar con c aca, como dije arriba. Es un buen exp, hay que pensar bien como graficarlo y que quede lindo, creo que es facil.

\subsection{Metodo de la Potencia}

\subsubsection{Representacion de la Matriz de Transicion (RODRI/FEDE)}
Este experimento lo pueden hacer directo o usando al PageRank. Si pueden, implementen todas las representaciones de matrices y luego comparen el tiempo de computo del producto N veces. Comparen la matriz normal vs el resto. Discutan que en páginas web la cantidad de vertices del grafo se va al carajo, pero para deportes es super acotada, asi que la eleccion de estructura no afecta tanto.

Aca podes argumentar que lo que domina al metodo de la potencia es la cantidad de productos, asi que no hace falta probar PageRank directo. Igual si queres metelo con pagerank de una, a fin de cuentas es lo mismo.

\subsubsection{Evolucion de la norma entre iteraciones}
Como va evolucionando la norma manhattan entre dos iteraciones sucesivas. Eje x, iteraciones, eje y, norma manhattan.

\subsubsection{Convergencia}
Aca tienen que calcular el vector posta, y luego tomar algun tipo de norma. En el eje x van a tener la cantidad de iteraciones, y en el eje y van a tener la norma de x* - $x_actual$.

\subsubsection{Eleccion del $x_0$}

Aca pongan que te conviene arrancar con una buena 'adivinanza' de la solucion, asi se acerca mas rapido. Muestren la cantidad de iteraciones a la convergencia (norma manhattan < epsilon) dependiendo de la distanciia de la solucion inicial a la solucion posta. Si arranco con la posta de una, converge de una. Si arranco con una sol asquerosa inicial, tarda mas iteraciones en cumplir nuestro epsilon.

Mostrar dos instancias, una donde arrarnco desde el valor inicial donde todos tienen 1/n y otra donde una tiene 1 y el resto 0, mostrar la cantidad de pasos y como evoluciona la norma.



\newpage
\section{Conclusiones}

Una vez que ya este todo lo leo y escribo esto bien a los pedos, incluyendo la caratula.
\newpage

\section{Apéndice A: Enunciado}
\input{appendixA.tex}
\section{Apéndice B: Código}
\subsection{matrix.h}
\lstinputlisting[language=C++, breaklines=true]{../src/src/matrix.h}
\subsection{eqsys.h}
\lstinputlisting[language=C++, breaklines=true]{../src/src/eqsys.h}
\subsection{buildSystem.cpp}
\lstinputlisting[language=C++, breaklines=true]{../src/src/buildSystem.cpp}

\end{document}