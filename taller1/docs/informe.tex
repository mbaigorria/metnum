\documentclass[10pt,a4paper]{article}
\usepackage[paper=a4paper, hmargin=1.5cm, bottom=1.5cm, top=3cm]{geometry}

\usepackage[utf8x]{inputenc}
\usepackage[spanish]{babel}

\usepackage{mathtools}
\usepackage{amsmath}
\usepackage{amsfonts}
\usepackage{amssymb}

\usepackage{xcolor}
\usepackage{listingsutf8}
\usepackage{booktabs}
\usepackage{hyperref}
\usepackage{multirow}

\usepackage{caption}
\usepackage{subcaption}

\usepackage{algorithm}
\usepackage[noend]{algpseudocode}

\usepackage{graphicx}
\usepackage{tikz}
\usepackage{relsize}
\usepackage{epstopdf}

\usepackage{chessboard}
\storechessboardstyle{6x6}{maxfield=h8}

\DeclarePairedDelimiter{\ceil}{\lceil}{\rceil}

%\let\NombreFuncion=\textsc
%\let\TipoVariable=\texttt

%\newcommand{\TipoFuncion}[3]{%
  %\NombreFuncion{#1}(#2) \ifx#3\empty\else $\to$ \res\,: \TipoVariable{#3}\fi%
%}

% set the default code style
\lstset{
    frame=tb, % draw a frame at the top and bottom of the code block
    tabsize=4, % tab space width
    showstringspaces=false, % don't mark spaces in strings
    numbers=left, % display line numbers on the left
    commentstyle=\color{green}, % comment color
    keywordstyle=\color{blue}, % keyword color
    stringstyle=\color{red} % string color
}

% mathy stuff
\newtheorem{theorem}{Theorem}[section]
\newtheorem{lemma}[theorem]{Lemma}
\newtheorem{proposition}[theorem]{Proposición}
\newtheorem{corollary}[theorem]{Corollary}

\newenvironment{proof}[1][Demostración]{\begin{trivlist}
\item[\hskip \labelsep {\bfseries #1}]}{\end{trivlist}}
\newenvironment{definition}[1][Definición]{\begin{trivlist}
\item[\hskip \labelsep {\bfseries #1}]}{\end{trivlist}}
\newenvironment{example}[1][Example]{\begin{trivlist}
\item[\hskip \labelsep {\bfseries #1}]}{\end{trivlist}}
\newenvironment{remark}[1][Remark]{\begin{trivlist}
\item[\hskip \labelsep {\bfseries #1}]}{\end{trivlist}}

\newcommand{\qed}{\nobreak \ifvmode \relax \else
      \ifdim\lastskip<1.5em \hskip-\lastskip
      \hskip1.5em plus0em minus0.5em \fi \nobreak
      \vrule height0.75em width0.5em depth0.25em\fi}

\title{Métodos Numéricos \\ Taller 1: Image Denoising}

\newcommand{\order}[1]{$\mathcal{O}(#1)$}

\begin{document}

%% cover page

\maketitle

\bigskip

\begin{table}[h]
\centering
\begin{tabular}{|l l l|}
\hline
Integrante       & \multicolumn{1}{c}{LU}     & Correo electrónico        \\ \hline
Martin Baigorria & \multicolumn{1}{c}{575/14} & martinbaigorria@gmail.com \\ 
Federico Beuter & 827/13                      & federicobeuter@gmail.com \\
Mauro Cherubini & 835/13                      & cheru.mf@gmail.com \\ 
Rodrigo Kapobel & 695/12                      & rok\_35@live.com.ar \\  \hline
\end{tabular}
\end{table}

\vfill

\begin{center}
\textbf{Reservado para la cátedra}
\end{center}
\begin{table}[h]
\centering
\begin{tabular}{|l|l|l|}
\hline
Instancia       & Docente & Nota \\ \hline
Primera entrega &         &      \\ \hline
Segunda entrega &         &      \\ \hline
\end{tabular}
\end{table}

\newpage
\tableofcontents
\newpage

% end cover page

\section{Condiciones para garantizar que una matriz tiene factorización LU}

\subsubsection{Factorización LU}

Dada una matriz A cuadrada de dimension $n \times n$, la factorización LU busca expresar la matriz A como producto de una matriz triangular inferior $L$ (con unos en su diagonal) y una matriz triangular superior $U$. Es decir, buscamos $L$ y $U$ tal que $A = LU$.

\subsubsection{Matriz Simétrica Definida Positiva}

\begin{definition}
Una matriz $A$ de dimension $n \times n$ se dice simétrica definida positiva (sdp) si es simétrica y a su vez definida positiva.
\end{definition}

\begin{definition}
Una matriz se dice simétrica cuando $A = A^t$.
\end{definition}

\begin{definition}
Una matriz se dice definida positiva cuando $\forall x \in \mathbb{R}^n / x \neq 0,$ $x^tA^tAx > 0$.
\end{definition}

\subsubsection{Condiciones para garantizar que una matriz tiene factorización LU}

Las siguientes condiciones garantizan que una matriz $A$ tenga factorización LU.

\begin{proposition}
A tiene todas sus submatrices principales NO singulares $\iff$ A tiene factorización LU.
\end{proposition}

\begin{proposition}
A tiene todos sus menores distintos de cero $\iff$ A tiene factorización LU.
\end{proposition}

\begin{proposition}
A es una matriz estrictamente diagonal dominante $\implies$ A tiene factorización LU. 
\end{proposition}

\section{¿Es cierto que si una matriz es inversible entonces tiene factorizacion LU?. Y si tengo una matriz que tiene factorizacion LU, ¿entonces es no singular? Demostrar o dar un contraejemplo.}

\subsection{¿Es cierto que si una matriz es inversible entonces tiene factorizacion LU?}

No, que A sea una matriz inversible es condición necesaria pero no suficiente. Lo que si vale es: \textit{A tiene todas sus submatrices principales NO singulares $\iff$ A tiene factorización LU.}
\\
\\
Como contra ejemplo al caso anterior tenemos:
\\

$
\hspace{10mm}
\left(
\begin{matrix}
1 & 1 & 0 \\
1 & 1 & 1 \\
0 & 1 & 1 \\
\end{matrix}\right)
$
\\
\\
Que es inversible, pero su submatriz principal de $2x2$ conformada integramente de unos, no lo es. Notemos que tras aplicar la primer iteración del algoritmo de Eliminación Gaussiana, habrá un 0 en la diagonal y un 1 por debajo en su misma columna, obligando a permutar la segunda y tercer fila. Por lo que para obtener la matriz LU, habrá que componer a A por una matriz de permutación, lo que nos deja una fatorización PLU. 
%Notese que hay un "\iffalse".
\iffalse
No es cierto que si una matriz es inversible entonces tiene factorizacion LU. Consideremos una matriz inversible donde es necesaria la eliminacion gausiana con pivoteo.

\begin{equation}
PAx = Pb
\end{equation}

Luego, dado que A es inversible y una permutacion de sus filas es simplemente un reordenamiento de funciones que no afecta las propiedades de la base, podemos llevar a cabo la siguiente factorizacion, donde L es triangular inferior y U es triangular superior.

\begin{equation}
PA = LU
\end{equation}

Las matrices de permutacion a su vez tienen la siguiente propiedad: $P^{-1} = P^t$. Por lo tanto, podemos escribir la matriz A como:

\begin{equation}
A = P^tLU
\end{equation}

Esto significa que la matriz inversible A solo puede escribirse como $LU$ $\iff$ la matriz de permutacion es la identidad. Es decir, si no son necesarias permutaciones para  llevar a cabo la eliminacion gausiana.

\textbf{No estoy seguro que este bien. Creo que hay que plantearlo por el lado de que asumo que A tiene LU tambien, y despues hago algo como A = $P^tLU$ y A = $L_2$ $U_2$, igualo, y llego al absurdo.}
\fi
\pagebreak

\subsection{Una matriz que tiene factorizacion LU es no singular?}

Una matriz que tiene factorizacion LU no necesariamente es no singular. Consideremos el siguiente contraejemplo:
\\

$
\hspace{10mm}
\underbrace{\left(
\begin{matrix}
0 & 0 \\
0 & 0 \\
\end{matrix}\right)}_{0}
=
\underbrace{\left(
\begin{matrix}
1 & 0 \\
0 & 1 \\
\end{matrix}\right)}_{L}
\times
\underbrace{\left(
\begin{matrix}
0 & 0 \\
0 & 0 \\
\end{matrix}\right)}_{U}$

La matriz nula no es inversible dado que $det(0)=0$.

\section{Verdadero o Falso}

Sea $A \in \mathbb{R}^{n \times n}$, estudiemos las siguientes proposiciones:

\subsection{$AA^t$ es una matriz simétrica}

Una matriz es simétrica si $A = A^t$ o lo que es lo mismo $A_{i,j} = A_{j,i}$ $\forall j,i \in [1..n]$, pues se tiene que $A_{i,j}^t = A_{j,i}$.

\begin{equation}
\left(AA^t\right)_{i,j} = \sum_{k=1}^{n} A_{i,k} \times A^t_{k,j} = \sum_{k=1}^{n} A_{i,k} \times A_{j,k} = \sum_{k=1}^{n} A^t_{k,i} \times A_{j,k} = \sum_{k=1}^{n} A_{j,k} \times A^t_{k,i} = \left(AA^t\right)_{j,i} 
\end{equation}

%Notese que hay un "\iffalse".
\iffalse
Una matriz es simétrica si $A = A^t$. Es decir, si para todo $j,i \in [1..n]$, $a_{i,j} = a_{j,i}$.

Sea $C = AA^t$ y $b_{i,j}$ los elementos de $A^t$. Al hacer el producto, en la posición $c_{i,j}$ nos queda:

\begin{equation}
c_{i,j} = \sum_{k=1}^{n} a_{i,k} \times b_{k,j}
\end{equation}

Por la definición de transpuesta $b_{i,j} = a_{j,i}$, entonces:

\begin{equation}
c_{i,j} = \sum_{k=1}^{n} a_{i,k} \times a_{j,k}
\end{equation}

Para que el producto sea simétrico entonces $c_{i,j} = c_{j,i}$, lo que efectivamente sucede si intercambiamos los indices en la ecuación 5. 
\fi
\subsection{Si $A$ es no singular, entonces $A^tA$ es SDP}

Una matriz es simétrica si y sólo si su traspuesta es simétrica. Por el inciso anterior $AA^t$ es simétrica, lo que ocurre si y sólo si $(AA^t)^t = A^tA$ es simétrica.
\\
\\
La matriz $A^tA$ es definida positiva si y sólo si satisface que $\forall x \in \mathbb{R}^n / x \neq 0,$ $x^tA^tAx > 0$.
\\
Dado $x$ como antes, llamemos $z:=Ax$. Para lo siguiente notemos que como $A$ es \textbf{no} singular y $ x \neq 0$ se tiene que $z \neq 0$.

Ahora como $<z,z> = \sum_{i=1}^{n} z_i \times z_i = \sum_{i=1}^{n} (z_i)^2 \geq 0$ ya que $z \in \mathbb{R}^n$, se tiene que $<z,z> = 0 \Leftrightarrow z = 0$; y como además sabemos que $z \neq 0$ entonces $ <z,z>  > 0$.

Luego $0 < <z,z> = z^tz = (Ax)^tAx = x^tA^tAx$, lo que implica que $A^tA$ es definida positiva.

\pagebreak

\section{Implementacion}

\subsection{CheckCondLU.m}

Si todos los menores principales de una matriz son no singulares, entonces la matriz tiene factorizacion LU.

Si todos los menores principales de una matriz tienen determinante positivo si y solo si la matriz tiene factorizacion de Cholesky.

\subsection{CheckFromLU.m}

Para verificar si la factorizacion de Cholesky es efectivamente valida, podemos computar nuevamente la matriz haciendo $LL'$ y luego comparando elemento a elemento con la matriz A. Sin embargo, debemos recordar que estamos trabajando bajo aritmética finita, por lo que debemos tener algun tipo de tolerancia al error.

\subsection{CholFromBlocks.m}

Utilizando bloques, encontramos las siguientes expresiones para computar, a partir de la factorizacion de Cholesky de $A_n$ y la matriz $A_{n+1}$, la factorizacion de Cholesky de la matriz $A_{n+1}$.

\begin{equation}
A_n = L_n L'_n
\end{equation}

\begin{equation}
f_{n+1} = l_{n+1} L'_n \implies l_{n+1} = f_{n+1} L_n^{'-1}
\end{equation}

Aqui la inversa de $l'_n$ existe dado que todos los menores principales son no singulares. $det(A) = det(LL') = det(L) \times det(L') > 0$.

\begin{equation}
a_{n+1,n+1} = l_{n+1} l'_{n+1} + l_{n+1,n+1}^2
\end{equation}

\begin{equation}
l_{n+1,n+1} = \sqrt{a_{n+1,n+1} - l_{n+1} l'_{n+1}}
\end{equation}

\end{document}