\section{Introduccion}

El 25 de Mayo de 2015 el diario The New York Times publico un articulo titulado "Years Later for Guillermo Vilas, He's Still Not the One", donde se repasa el rendimiento del tenista argentino durante los años 1975/1976 y se discute el calculo del ranking de la ATP en ese momento. Aunque hoy en día Vilas es un icono del tenis argentino, nunca logró estar en la cima del ranking de la ATP.

\begin{figure}[H]
  \centering
  \includegraphics[scale=6]{images/nyt}
  \caption{Guillermo Vilas after winning a tournament in Stockholm in 1975. A journalist has asserted that Vilas deserved to be ranked No. 1 during that year. }
\end{figure}

En 2016, un grupo de investigadores y periodistas deportivos argentinos decidieron analizar el ranking de la ATP en 1975 y 1977 para determinar si Vilas debió haber sido número 1. Dado que los rankings no se actualizaban constantemente en ese momento, los investigadores mostraron que de haberse actualizado de forma periódica, Vilas hubiese sido número 1 por durante 7 semanas entre 1975 y 1977.

Existen precedentes donde se actualizó un ranking de tenis de forma retroactiva. Este es el caso de la WTA, que determinó que Evonne Goolagong Cawley debió haber sido número 1 por dos semanas en 1976. Por esta razón el grupo de investigación argentino considera que revisar estos rankings no es un esfuerzo en vano. Cuando buscábamos los datos de la ATP entre 1975/1977, uno de los investigadores de este equipo que contactamos nos comento: \say{Es interesante tu decisión de indagar sobre el tema. Tal vez no estás al tanto del trabajo y lucha que estamos realizando contra la ATP, por el ranking de los 70 en el que perjudicaron a Vilas y muchísimos otros jugadores.}.

En ese momento, el calculo del ranking de la ATP era bastante rudimentario: \say{It was a system based on an average of a player's results, and it often rewarded top players who played fewer tournaments. Vilas was a workhorse, which is how he managed not to reach
No. 1 in the ATP rankings in 1977, when he won the French Open, the United States Open and 14 other tournaments.} \cite{nyt}.

Los métodos para calcular rankings no solo son relevantes para definir las posiciones de equipos y jugadores en eventos deportivos, sino que aparecen constantemente en todo tipo de situaciones donde se debe imponer algún tipo de orden. Este es el caso por ejemplo de los concursos docentes, donde se ponderan los diferentes antecedentes para decidir cual es el candidato \textit{idoneo} para el puesto.

Otro caso sumamente relevante en cuanto algoritmos de rankeo es el de los motores de búsqueda. Los motores de búsqueda deben encontrar alguna forma de ordenar de forma relevante los sitios web que están relacionados con una consulta. El caso icónico es el de Google con su algoritmo PageRank. Los buscadores antes de 1990 eran sumamente rudimentarios, utilizaban algoritmos de rankeo vulnerables en el sentido que podían ser manipulados y no se explotaba gran parte de la estructura de la web. Esta fue una de las razones por las cuales una consulta no siempre devolvía resultados relevantes. Este fue el caso por ejemplo de algunos buscadores en ese momento como Yahoo! Search o AltaVista.

\begin{figure}[H]
  \centering
  \includegraphics[scale=0.4]{images/altavista}
  \caption{Sitio Web de Altavista, ano 1999.}
\end{figure}

El clásico paper de Brin y Page, \say{The anatomy of large-scale hypertextual Web search engine.} \cite{Brin1998} explica brevemente el origen del motor de búsqueda de Google y del algoritmo PageRank. La idea es básicamente la siguiente, en primer lugar se implementa un crawler distribuido para poder solicitar y armar el grafo de la web. Las palabras de cada sitio son indexadas y guardadas en una base de datos. Al llegar una consulta al buscador, un programa busca la consulta en los indices de páginas. De esta forma llegamos a un conjunto de páginas que están relacionadas con la consulta. Luego, antes de devolverle al usuario los resultados, estas páginas son ordenadas utilizando el famoso algoritmo PageRank. Este algoritmo se basa en la idea de que para medir la relevancia de un sitio se puede usar como proxy la cantidad de sitios que tienen un link al mismo. Para evitar que un usuario malintencionado manipule los resultados del mismo, la relevancia otorgada por un sitio web que linkea a otro es proporcional a su propia relevancia e inversamente proporcional a la cantidad de links (o grado de salida) del mismo.

El presente trabajo práctico tendrá como objetivo implementar el algoritmo PageRank para luego utilizarlo para generar rankings de todo tipo, ya sea para ordenar la relevancia de páginas webs o generar rankings deportivos. PageRank es un algoritmo que basa su ranking en la idea abstracta del navegante aleatorio. Este problema en general se modela con Cadenas de Markov, y en ultima instancia consiste en encontrar el autovector de una matriz de transiciones, que es similar a una matriz de adyacencia en teoría de grafos. A priori esto puede sonar complicado, pero luego mostraremos que en realidad es bastante simple y elegante.

Debido a su relevancia, PageRank es un algoritmo que ha sido ampliamente estudiado en la literatura. Una muy buena introducción teórica se puede encontrar en el trabajo de Bryan y Leise \cite{Bryan2006}. Otros autores como Kamvar et al. \cite{Kamvar2003} han buscado otros enfoques y métodos para poder acelerar este algoritmo. La idea es encontrar una forma eficiente de poder computar este modelo, calibrando sus diferentes parámetros de modelado y convergencia para lograr un orden relevante. Otros autores como Govan et al. han buscado modelar la matriz de transiciones para aplicar este algoritmo en eventos deportivos \cite{Govan2008}. Es importante mencionar que en este caso la complejidad temporal de los algoritmos utilizados para calcular PageRank no tendra tanta relevancia dado que la cantidad de equipos es acotada. La forma de calibrar el modelo tambien ha sido amplicamente estudiada, en especial para el caso de las páginas web. Un estudio muy interesante es el Constantine et al. de Microsoft Research, donde utilizando los registros del toolbar de Internet Explorer y un modelo basado en la distribucion Beta generalizan la matriz de transiciones para poder modelar un parametro de teletransportacion variable \cite{TeleParam}.

Una vez planteado el procedimiento para computar el PageRank, experimentaremos con la complejidad temporal de los métodos implementados y evaluaremos los diferentes parámetros a calibrar. Finalmente concluiremos si según el algoritmo PageRank y nuestro modelado Vilas efectivamente debió haber estado en la punta del ATP en periodo 1975/1977. En caso afirmativo, sin dudas nos comunicaremos con la ATP.

\begin{figure}[H]
  \centering
  \includegraphics[scale=0.5]{images/vilasaprueba}
  \caption{Guillermo Vilas apoya este TP.}
\end{figure}