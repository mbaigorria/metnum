\section{Conclusiones}

Los algoritmos de rankeo son sumamente relevantes. Se utilizan en todo tipo de situaciones donde es necesario asignar un orden, ya sea entre personas, equipos o países. En el presente trabajo se decidió analizar el algoritmo PageRank. Este algoritmo fue un algoritmo icónico que marco el inicio de lo que hoy conocemos como Google en 1998. Hoy en día se sigue utilizando el algoritmo original y muchas variantes del mismo en la practica.

Como ya hemos mostrado, el problema se reduce a encontrar un autovector de norma unitaria correspondiente al autovalor 1 del sistema, es decir, buscamos $x$ tal que $Ax = x$, donde $A$ es la matriz de transicion y $x$ representa un ranking valido. Una de las primeras cuestiones que nos encontramous al tener que aplicar el algoritmo en la practica es el modelado de la matriz de transiciones. Se debe demostrar que nuestro modelado cumple con que la matriz es estocastica positiva por columnas


Finalmente, concluimos que si la ATP hubiese utilizado PageRank, efectivamente Vilas hubiese sido líder del ranking de la ATP en el año 1977, convirtiéndose en el primer argentino en lograr semejante hazaña.