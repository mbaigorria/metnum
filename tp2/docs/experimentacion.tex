\section{Experimentación}

\textbf{ESTA PARTE ES DE FEDE. TENES QUE COMPLETAR TODOS LOS EXPERIMENTOS QUE LISTO A CONTINUACION, FIJATE SI SE TE OCURRE ALGUNO, LO QUE TE TIRO SON UNAS PUNTAS, EL RESTO SE TE OCURRE A VOS ;) MIENTRAS RODRI TERMINA DE IMPLEMENTAR TODO Y VER COMO CARAJO MODELAR BIEN EL TENIS, VOS DISENIA LOS EXPERIMENTOS O FIJATE QUE PODES IR HACIENDO. ACA NO HAY QUE GENERAR INSTANCIAS LOCAS!}

Experimentos para Paginas Web
\begin{itemize}
\item Comparacion PageRank vs In-Deg. Tiempo de ejecucion, relevancia de los resultados.
\item Cuanto tarda el pagerank a medida que aumenta la cantidad de vertices del grafo?
\end{itemize}

Experimentos para competencias deportivas
\begin{itemize}
\item GeM y algun otro metodo estandar, onda victoria 3 puntos derrota 1, lo que se te ocurra para tenis.
\end{itemize}

Experimentos para el PageRank en si\begin{itemize}
\item Como va evolucionando la norma manhattan entre dos iteraciones sucesivas. Eje x, iteraciones, eje y, norma manhattan.
\item Mostrar dos instancias, una donde arrarnco desde el valor inicial donde todos tienen 1/n y otra donde una tiene 1 y el resto 0, mostrar la cantidad de pasos y como evoluciona la norma.
\item tiempo de computo en funcion de size del grafo
\item probar relevancia a medida que cambias ese valor = 0.85, creo que c.
\end{itemize}

Leer la parte de experimentos del enunciado....

Los siguientes exp se me ocurrieron o los saque mientras leia el paper de Bryan y Leise, te los pongo aca:

1. Caso particular chiquito, pagina 3. Fijate el parrafo que arranca en A simple apprroach...... y despues This approach ignores that... La idea es armar el mismo grafo y mostrar el mismo ejemplo jaja
2. Pagina 5, ejercicio 1. La idea es que plantees un caso de un tipo que quiere manipular el ranking, mostra que aunque agregues miles de nuevas paginas apuntando no podes hacer demasiado, hacelo en funcion de la cantidad de paginas que agregas?
3. Pagina 6.... This is the ultimately egalitarian case: the only... blah. La idea es jugar con c aca, como dije arriba. Es un buen exp, hay que pensar bien como graficarlo y que quede lindo, creo que es facil.
4. Para todo $x_0$ sabes que converge, eso, analiza como funciona eso, pasos para que converja en funcion de $x_0$.
5. Si llegamos, comparar dos versiones del metodo de la potencia! Esto seria genial.

\textbf{Fede, cuando leas esto no te preocupes. Lee lo que escribi arriba, los enunciados y solo el paper de Bryan and Leise. Con eso te re alcanza para hacer lo que tenes que hacer. Lo que te recomiendo es que una vez que leas eso ordenes bien mis ideas y decidas bien que queres hacer. Fijate si podes revisar el codigo de Rodri!}

