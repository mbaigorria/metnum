\section{Experimentación}

\subsection{PageRank}
\subsubsection{Complejidad}
tiempo de computo en funcion de size del grafo, eje x, cantidad de sitios web, eje y, tiempo en ms a convergencia.

\subsubsection{Casos Patologicos}
Caso particular chiquito, pagina 3. Fijate el parrafo que arranca en A simple apprroach...... y despues This approach ignores that... La idea es armar el mismo grafo y mostrar el mismo ejemplo jaja

\subsection{Paginas Web}

\subsubsection{Comparacion PageRank vs In-Deg (RODRI)}
Comparar solo los rankings, nada de complejidad. Podes mencionar que In-Deg usa un algoritmo \order{n \times log(n)}, pero nada mas. Comparar top 10 con los dos y discutir diferenciias.

\subsubsection{Manipulacion}
Pagina 5, ejercicio 1. La idea es que plantees un caso de un tipo que quiere manipular el ranking, mostra que aunque agregues miles de nuevas paginas apuntando no podes hacer demasiado, hacelo en funcion de la cantidad de paginas que agregas?

Se puede manipular entonces o no? Agarra, en el eje x pone cantidad de sitios web que apuntan solamente al sitio u que le quiero subir el ranking, y en el eje y el ranking de ese sitio. Fijate que aumenta, y fijate si podes hacer algun tipo de curva de nivel con c (cuanto mayor c, mas manipulable es la cosa). Citar el paper de Sergei y Brin, que dicen que hacen promedios de muchas cosas en la practica para evitar este problema. Usan muchos criterios promediados.

\subsection{Ranking ATP}

Empezemos con el apartado que seguro el lector más esperaba de todo el t.p..Vilas fue o no 1ro entre 1975 y 1977? Esta pregunta podemos contestarla. Pero previo a esto, necesitamos poner al lector al tanto de la situación. Empecemos viendo el ranking oficial de la época.


En 1975, Vilas llegó a la primera final de un Grand Slam, el Roland Garros, en donde fue derrotado por el sueco Björn Borg por 2-6 3-6 4-6
y cuartos de final de Wimbledon.


Este es el top 10 para 1975 segun la ATP, donde Guillermo Vilas se hubica 2do:

\begin{eqnarray*}
173 & Jimmy & Connors \\
127 & Guillermo & Vilas \\
34 & Bjorn & Borg \\
19 & Arthur & Ashe \\
225 & Manuel & Orantes \\
210 & Ken & Rosewall \\
144 & Ilie & Nastase \\
180 & John & Alexander \\
318 & Roscoe & Tanner \\
309 & Rod & Laver 
\end{eqnarray*}

En 1976 la ATP hubica a Vilas 6to dentro del top 10:

\begin{eqnarray*}
173 & Jimmy & Connors \\
34 & Bjorn & Borg \\
144 & Ilie & Nastase \\
225 & Manuel & Orantes \\
288 & Raul & Ramirez \\
127 & Guillermo & Vilas \\
1 & Adriano & Panatta \\
134 & Harold & Solomon \\
87 & Eddie & Dibbs \\
41 & Brian & Gottfried 
\end{eqnarray*}

En 1977 Vilas conquistó 17 torneos (récord todavía vigente), se consagró en Roland Garros frente a Brian Gottfried por 6-0 6-3 6-0 y el US Open frente a Jimmy Connors por 2-6 6-3 7-6 6-0, fue finalista de Australia donde cayó frente a Roscoe Tanner por 3-6 3-6 3-6, logró una seguidilla de 50 partidos consecutivos sin conocer la derrota y, durante esos doce meses, ganó 145 de los 159 encuentros que jugó (91,1\%) hubicandose como 2do en el ranking de la ATP de finales de ese año.
Según el ranking de la ATP, el mejor del año fue el norteamericano Jimmy Connors, batido por Vilas en la final del US Open (detalle a tener en cuenta), que en ese 1977 no obtuvo ningún torneo de Grand Slam y ganó ocho torneos, menos de la mitad de los logrados por Vilas.  

Veamos el top 10 oficial para este año:

\begin{eqnarray*}
173 & Jimmy & Connors \\
127 & Guillermo & Vilas \\
34 & Bjorn & Borg \\
371 & Vitas & Gerulaitis \\
41 & Brian & Gottfried \\
87 & Eddie & Dibbs \\
225 & Manuel & Orantes \\
288 & Raul & Ramirez \\
144 & Ilie & Nastase \\
81 & Dick & Stockton 
\end{eqnarray*}

Todos los años tienen como lider indiscutido al estadounidense Jimmy Connors.

\subsubsection{Ranking ATP oficial vs. Ranking PageRank vs. Sort por diferencia de victorias/derrotas}

Introducimos aquí un algoritmo de rankeo con un criterio de ordenamiento basado en victorias/derrotas Además, si hay empate, define por diferencia de puntos, pero para el caso de tenis y como esta definido el sistema de puntos del mismo no tiene ninguna relevancia. Solo nos servirá para poder hacer un análisis cualitativo de las virtudes de pagerank sobre algoritmos más básicos.

Veamos que obtuvimos en cada año con este algoritmo, empezando por 1975:

\begin{eqnarray*}
19 & 96 & 18 \\
225 & 90  & 20 \\
127 & 86 & 18 \\
173 & 79 & 11 \\
144 & 88 & 22 \\
34 & 82 & 17 \\
288 & 70 & 28 \\
180 & 65 & 23 \\
318 & 64 & 24 \\
155 & 60 & 21 \\
\end{eqnarray*}

Podemos ver que Vilas no aparece segundo, si no tercero. Connors fue desplazado al 4to lugar y en primer lugar aparece Arthur Ashe quien ocupaba el lugar que Connors ocupa ahora. Podemos ver varios cambios relacionados al top 10 oficial. Todos ellos particularmente relacionados al hecho de que como se habia anticipado la cantidad de victorias sobre derrotas conformaria otro ranking diferente en el cual no importa exactamente que clase de victorias a conseguido o derrotas sufrido un determinado participante. Podemos ver que Vilas con más victorias sobre derrotas que Connors, se encontraba por debajo de él en el ranking oficial. Esto se debe al sistema de puntos manejados por el ATP donde se suma más puntaje cuanto mas avancemos en un torneo y cuantas más finales ganemos y por la jerarquía de ese torneo. 

Observemos el ranking de 1976:

\begin{eqnarray*}
173 & 86 & 9 \\
127 & 83 & 20 \\
144 & 73 & 15 \\
288 & 91 & 33 \\
225 & 76 & 19 \\
87 & 83 & 28 \\
34 & 57 & 12 \\
318 & 71 & 27 \\
380 & 75 & 32 \\
134 & 70 & 27 \\
\end{eqnarray*}


Vemos que Vilas avanzó del 6to lugar al 2do puesto!! que era ocupado por una bestia como Bjorn Borg y que fue desplazado al 7mo. Si analizamos un poco los partidos veremos que Vilas fué derrotado por Bjorn Borg 3 veces (Winbledon, Dallas WCT, Sao Paulo WCT) mientras que Vilas nunca pudo derrotarlo, Contra Connors se enfrentó una vez y fue derrotado (US Open) al igual que Bjorn Borg que se enfrentó en 3 ocaciones y cayó en la misma cantidad...esto nuevamente es un indicador de que determinadas victorias son más importantes que otras.

Veamos que sucede en 1977:

\begin{eqnarray*}
127 126 14 
41 105 22 
34 71 8 
173 69 15 
87 77 28 
371 60 15 
225 65 25 
134 64 27 
288 61 24 
380 66 30 
\end{eqnarray*}


...humo humo humo. mostrar que vilas en 1975 y 1976 no le gano a nadie groso y por eso no esta primero, eso en consecuencia es que perdio en instancias mas o menos decisivas contra esos rivales grosos y por eso mantiene un lugar alto en el ranking..ademas todos ellos le ganaron a muchos que a su vez le ganaron muchos. 
Mostrar que esto no sucede para 1977 donde ampliamente vilas se los viola porque gano todo y ademas y lo mas importante le gano a jimmy neutron y por eso rankeo mucho mas puntaje. que a su vez ellos dos son los unicos con 0.30 y pico en el ranking y que los anteriores estan mas parejos porque todos estaban mas parejos (?) 

\subsubsection{Eleccion del factor de 'teletransportacion' c (RODRI)}
probar relevancia a medida que cambias ese valor = 0.85, creo que c.

Citar paper de google, que usan 0.85. Discutir que si c es uno, ignoras la estructura del grafo al hacer el ranking, todos rankean igual.

Pagina 6.... This is the ultimately egalitarian case: the only... blah. La idea es jugar con c aca, como dije arriba. Es un buen exp, hay que pensar bien como graficarlo y que quede lindo, creo que es facil.

\subsection{Metodo de la Potencia}

\subsubsection{Representacion de la Matriz de Transicion (RODRI/FEDE)}
Este experimento lo pueden hacer directo o usando al PageRank. Si pueden, implementen todas las representaciones de matrices y luego comparen el tiempo de computo del producto N veces. Comparen la matriz normal vs el resto. Discutan que en paginas web la cantidad de vertices del grafo se va al carajo, pero para deportes es super acotada, asi que la eleccion de estructura no afecta tanto.

Aca podes argumentar que lo que domina al metodo de la potencia es la cantidad de productos, asi que no hace falta probar PageRank directo. Igual si queres metelo con pagerank de una, a fin de cuentas es lo mismo.

\subsubsection{Evolucion de la norma entre iteraciones}
Como va evolucionando la norma manhattan entre dos iteraciones sucesivas. Eje x, iteraciones, eje y, norma manhattan.

\subsubsection{Convergencia}
Aca tienen que calcular el vector posta, y luego tomar algun tipo de norma. En el eje x van a tener la cantidad de iteraciones, y en el eje y van a tener la norma de x* - $x_actual$.

\subsubsection{Eleccion del $x_0$}

Aca pongan que te conviene arrancar con una buena 'adivinanza' de la solucion, asi se acerca mas rapido. Muestren la cantidad de iteraciones a la convergencia (norma manhattan < epsilon) dependiendo de la distanciia de la solucion inicial a la solucion posta. Si arranco con la posta de una, converge de una. Si arranco con una sol asquerosa inicial, tarda mas iteraciones en cumplir nuestro epsilon.

Mostrar dos instancias, una donde arrarnco desde el valor inicial donde todos tienen 1/n y otra donde una tiene 1 y el resto 0, mostrar la cantidad de pasos y como evoluciona la norma.


