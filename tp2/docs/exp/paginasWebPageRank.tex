\subsubsection{Comparacion PageRank vs In-Deg (RODRI)}
Comparar solo los rankings, nada de complejidad. Podes mencionar que In-Deg usa un algoritmo \order{n \times log(n)}, pero nada mas. Comparar top 10 con los dos y discutir diferenciias.