\subsection{Ranking ATP}

Empezemos con el apartado que seguro el lector más esperaba de todo el t.p..Vilas fue o no 1ro entre 1975 y 1977? Esta pregunta podemos contestarla. Pero previo a esto, necesitamos poner al lector al tanto de la situación. Empecemos viendo los rankings oficiales de la época.

En 1975, Vilas llegó a la primera final de un Grand Slam, el Roland Garros, en donde fue derrotado por el sueco Björn Borg
y cuartos de final de Wimbledon.

Este es el top 10 para 1975 segun la ATP, donde Guillermo Vilas se hubica 2do:

\begin{table}[H]
\label{my-label}
\begin{tabular}{ll}
\hline
id  & nombre \\ \hline
173 & Jimmy Connors \\
127 & Guillermo Vilas \\
34  & Bjorn Borg \\
19  & Arthur Ashe \\
225 & Manuel Orantes \\
210 & Ken Rosewall \\
144 & Ilie Nastase \\
180 & John Alexander \\
318 & Roscoe Tanner \\
309 & Rod Laver \\ \hline
\end{tabular}
\centering
\caption{Ranking ATP 1975}
\end{table}

En 1976 la ATP ubica a Vilas 6to dentro del top 10:

\begin{table}[H]
\label{my-label}
\begin{tabular}{ll}
\hline
id  & nombre \\ \hline
173 & Jimmy Connors \\
34  & Bjorn Borg \\
144 & Ilie Nastase \\
225 & Manuel Orantes \\
288 & Raul Ramirez \\
127 & Guillermo Vilas \\
1   & Adriano Panatta \\
134 & Harold Solomon \\
87  & Eddie Dibbs \\
41  & Brian Gottfried \\ \hline 
\end{tabular}
\centering
\caption{Ranking ATP 1976}
\end{table}

En 1977 Vilas se ubica segundo, por debajo de Connors. Recordemos que esto hasta el día de hoy sigue creando polémicas debido a los muy buenos resultados obtenidos por Vilas en aquel año comparados con los del estadounidense.
Veamos el top 10 oficial para este año:

\begin{table}[H]
\label{my-label}
\begin{tabular}{ll}
\hline
id  & nombre \\ \hline
173 & Jimmy Connors \\
127 & Guillermo Vilas \\
34  & Bjorn Borg \\
371 & Vitas Gerulaitis \\
41  & Brian Gottfried \\
87  & Eddie Dibbs \\
225 & Manuel Orantes \\
288 & Raul Ramirez \\
144 & Ilie Nastase \\
81  & Dick Stockton \\ \hline 
\end{tabular}
\centering
\caption{Ranking ATP 1977}
\end{table}

Todos los años tienen como lider indiscutido al estadounidense Jimmy Connors.

\subsubsection{Ranking ATP oficial vs. Ranking PageRank vs. Sort por diferencia de victorias/derrotas}

Introducimos aquí un algoritmo de rankeo con un criterio de ordenamiento basado en victorias/derrotas, que además, si hay empate define por diferencia de puntos, aunque para el caso de tenis y por como esta definido el sistema de puntos del mismo no tiene ninguna relevancia. Nos servirá para poder hacer un análisis cualitativo de las virtudes de PageRank sobre algoritmos más básicos y comparar estos resultados con el ranking oficial y poder obtener así alguna conclusión sobre los mótivos principales de la investigación.

Veamos que obtuvimos en cada año con este algoritmo, empezando por 1975:

\begin{table}[H]
\label{my-label}
\begin{tabular}{lll}
\hline
id  & PG & PP \\ \hline
19  & 96 & 18 \\
225 & 90 & 20 \\
127 & 86 & 18 \\
173 & 79 & 11 \\
144 & 88 & 22 \\
34  & 82 & 17 \\
288 & 70 & 28 \\
180 & 65 & 23 \\
318 & 64 & 24 \\
155 & 60 & 21 \\ \hline 
\end{tabular}
\centering
\caption{Ranking 1975 PG(partidos ganados) / PP(partidos perdidos)}
\end{table}

Podemos ver que Vilas no aparece segundo, si no tercero. Connors fue desplazado al 4to lugar y en primer lugar aparece Arthur Ashe quien ocupaba el lugar que Connors ocupa ahora. Podemos ver varios cambios relacionados al top 10 oficial.

Todos ellos particularmente relacionados al hecho de que como se habia anticipado la cantidad de victorias sobre derrotas conformaria otro ranking diferente en el cual no importa exactamente que clase de victorias ha conseguido o derrotas sufrido un determinado participante. Podemos ver que Vilas con más victorias sobre derrotas que Connors, se encontraba por debajo de él en el ranking oficial. Esto se debe al sistema de puntos manejados por el ATP donde se suma más puntaje cuanto más avancemos en un torneo y cuantas más finales ganemos y por la jerarquía de ese torneo. Claramente no alcanza con ganar un solo torneo importante, para sumar se tiene que jugar.

Observemos el ranking de 1976:

\begin{table}[H]
\label{my-label}
\begin{tabular}{lll}
\hline
id  & PG & PP \\ \hline
173 & 86 & 9 \\
127 & 83 & 20 \\
144 & 73 & 15 \\
288 & 91 & 33 \\
225 & 76 & 19 \\
87  & 83 & 28 \\
34  & 57 & 12 \\
318 & 71 & 27 \\
380 & 75 & 32 \\
134 & 70 & 27 \\ \hline 
\end{tabular}
\centering
\caption{Ranking 1976 PG(partidos ganados) / PP(partidos perdidos)}
\end{table}

Vemos que Vilas avanzó del 6to lugar al 2do puesto que era ocupado por Bjorn Borg, que a su vez, fue desplazado a la 7ma posición.

Veamos que sucede en 1977:

\begin{table}[H]
\label{my-label}
\begin{tabular}{lll}
\hline
id  & PG & PP \\ \hline
127 & 126 & 14 \\
41 & 105 & 22 \\
34 & 71 & 8 \\
173 & 69 & 15 \\
87 & 77 & 28 \\
371 & 60 & 15 \\
225 & 65 & 25 \\
134 & 64 & 27 \\
288 & 61 & 24 \\
380 & 66 & 30 \\ \hline 
\end{tabular}
\centering
\caption{Ranking 1977 PG(partidos ganados) / PP(partidos perdidos)}
\end{table}

Ok. Tomemoslo con calma. Vilas aparece primero pero esto no es indicador absoluto de que la ATP cometió un error, debido a la falta de información que provee. Aunque si nos da indicios de lo que realmente pasó.

La diferencia de partidos ganados sobre perdidos con respecto a cualquier otro competidor es bastante significativa. Podemos ver que Connors fué desplazado al 4to lugar. Segundo se ubica Brian Gottfried, que durante ese año tuvo un gran desempeño, entre los que se encuentra su victoria sobre Vilas en la final del Roland Garros. 

A priori...cantidad de victorias está relacionada con cantidad de puntos, pero esto no es una regla general y depende como mencionamos del sistema de puntajes. Para poder asemejar a la relevancia que la ATP le da a los torneos tenemos que utilizar un algoritmo que aproveche esa carácteristica lo mejor posible. Para esto haremos uso del modelo GeM y analizaremos sus resultados. Esperamos que los mismos se parezcan al ranking oficial, obviamente, con ciertas variaciones.

Utilizaremos los siguientes parámetros para generar los 3 rankings:

$c$ = 0.85
$\delta$ = 0.00001

Además, como indicamos en el diseño del sistema, usaremos una matriz de personalización uniforme, dado que no nos interesa que influyan sobre los resultados ninguna información estadistica de un torneo anterior, por el simple hecho de que evaluamos a los jugadores desde cero cada año.

Avancemos sobre los resultados, comenzando con 1975:

\begin{table}[H]
\label{my-label}
\begin{tabular}{ll}
\hline
id  & puntaje \\ \hline
19  & 0.033172 \\
34  & 0.030089 \\
225 & 0.026483 \\
144 & 0.026254 \\
127 & 0.023572 \\
173 & 0.021918 \\
180 & 0.017379 \\
318 & 0.015914 \\
309 & 0.011614 \\
210 & 0.011055 \\ \hline 
\end{tabular}
\centering
\caption{Ranking 1975 GeM}
\end{table}

Vemos que Vilas se ubica en el 4to puesto, Connors fué desplazado al 5to puesto y por arriba se ubican Ashe en primer puesto y Borg en segundo lugar. 

Si analizamos los partidos podremos saber que Ashe se enfrentó en una sola oportunidad a Connors y logró derrotarlo, en lo que fué la final Wimbledon. Además se enfrento en 7 oportunidades a Borg y logró vencerlo en 4 (Wimbledon, Barcelona WCT, Dallas WCT, Munich WCT), siendo Wimbledon el más importante de los 7 enfrentamientos.

Borg obtuvo la final de Roland Garros frente a Vilas como máximo logro. 

Mientras tanto Manuel Orantes hizo lo propio para obtener el 3er lugar derrotando a Vilas en las 4 oportunidades que se encontraron y derrotando a Connors en su único enfrentamiento. 
No es una tarea fácil determinar todas estas congruencias pero con un simple vistazo a los partidos jugados y los torneos en los que se enfrentaron parece claro que el ranking es elocuente.

Veamos el ranking de 1976:

\begin{table}[H]
\label{my-label}
\begin{tabular}{ll}
\hline
id  & puntaje \\ \hline
173 & 0.033300 \\
144 & 0.031773 \\
288 & 0.029343 \\
87  & 0.026328 \\
41  & 0.025002 \\
134 & 0.024189 \\
127 & 0.024006 \\
34  & 0.023806 \\
225 & 0.020587 \\
1   & 0.014804 \\ \hline 
\end{tabular}
\centering
\caption{Ranking 1976 GeM}
\end{table}

No parace haber mucho que analizar con respecto al ranking oficial. Vilas se encuentra una posición más abajo. Connors sigue siendo lider indiscutido y la otras posiciones relativas no han cambiado demasiado. Podriamos resaltar el caso particular de Borg que no tuvo malos resultados durante ese año, destacando su final ganada contra Ilie Nastase en Wimbledon, una final perdida en el US Open contra Connors y Cuartos de final en un Roland Garros. Pero si miramos el mismo ranking pero por diferencia de ganados/perdidos veremos que Borg se encuentra casi en la misma posición debido a la baja diferencia de partidos ganados sobre perdidos.

Concluyamos con el análisis de 1977: 

\begin{table}[H]
\label{my-label}
\begin{tabular}{ll}
\hline
id  & puntaje \\ \hline
127 & 0.037046 \\
41  & 0.035463 \\
34  & 0.029461 \\
173 & 0.027319 \\
134 & 0.024189 \\
87  & 0.023230 \\
81  & 0.021816 \\
371 & 0.017483 \\
288 & 0.016667 \\
225 & 0.016318 \\ \hline 
\end{tabular}
\centering
\caption{Ranking 1977 GeM}
\end{table}

Por una diferencia significativa, Vilas desplaza del primero puesto a Connors para quedarse el con el trono. Pero hay una razón bastante justificada para que esto suceda.
En 1977 Vilas conquistó 17 torneos (récord todavía vigente), se consagró en Roland Garros frente a Brian Gottfried por 6-0 6-3 6-0 y el US Open frente a Jimmy Connors por 2-6 6-3 7-6 6-0, fue finalista de Australia donde cayó frente a Roscoe Tanner por 3-6 3-6 3-6, logró una seguidilla de 50 partidos consecutivos sin conocer la derrota y, durante esos doce meses, ganó 145 de los 159 encuentros que jugó (91,1\%).
Jimmy Connors, en 1977 no obtuvo ningún torneo de Grand Slam y ganó ocho torneos, menos de la mitad de los logrados por Vilas. 
Parece lógico que Connors incluso esté unos escalones más abajo. Además el ranking por partidos ganados/perdidos para los primeros 4 es idéntico. Esto no puede hacer más que confirmar que Vilas fué el primero indiscutido en 1977 por una amplia diferencia, sobre todo por el bajo rendimiento de Connors durante ese año.

Como comentamos al principio, la cantidad de partidos ganados y los rivales derrotados son factores importantes a la hora de generar el ranking con GeM. Si la ATP hubiese cambiado a este método y fuera retroactivo, hoy el número 1 indiscutido de 1977 sería sin lugar a dudas Guillermo Vilas.

