\section{Conclusiones}

Los metodos de interpolacion ocupan un lugar muy relevante en muchas areas. Estos tienen aplicaciones muy variadas, en este trabajo ahondamos en una de ellas en particular, que fue la interpolacion de video. Particularmente en este trabajo se analizaron tres tecnicas distintas para interpolacion.

Como pudimos ver en la introduccion y el desarrollo, los diferentes metodos de interpolacion vistos son sumamente aptos para ser implementados de manera iterativa en software. Nearest Neighbor puede ser implmentado facilmente con una ecuacion de distancia, para la interpolacion lineal alcanza con plantear la recta entre los puntos, y por ultimo para los Splines Cubicos para obtener los coeficientes podemos plantear y resolver un sistema de ecuaciones lineal, en este ultimo caso vimos tambien que ademas el sistema podia resolverse con factorizacion LU sin pivoteo.

Una de las problematicas que nos encontramos, fue decidir como medir el error de cada uno de los metodos. Debido a que no tenemos una funcion con la cual comparar, tuvimos que ser un poco creativos a la hora de las mediciones, decidiendo tomar videos grabados a alta velocidad, quitar cuadros y regenerandolos mediante los metodos de interpolacion. El primer problema que encontramos fue elegir un video apropiado, esto fue dificil ya que los videos a alta velocidad disponibles en internet usualmente tenian una calidad bastante baja, tras mucho buscar finalmente encontramos videos de calidad apropiada. Una vez que ya teniamos los datos apropiados, los resultados de la experimentacion se ajustaron bastante a lo que esperabamos, el metodo de Splines resulto ser bastante superior en calidad, sin embargo esto tuvo un costo sumamente alto en tiempo de procesamiento, los tiempos resultantes fueron muy altos para Splines respecto a los otros metodos.

Un punto que consideramos importante y diferente de los otros trabajos realizados, es que la interoplacion de videos no es algo que se analiza unicamente de manera numerica, a fin de cuentas se esta buscando algo que sea fluido y agradable a la vista, esto lo pudimos al ver que los numeros arrojados por el analisis numerico no aportaron informacion suficiente para poder calificar los metodos, el resultado final dependio mayormente del analisis cualitativo. Desde el punto de vista del planteo de experimentos, nos vimos bastante limitados por la forma de manejar la informacion, particularmente nos topamos con problemas de espacio para poder almacenar los videos generados en texto plano. De los resultados obtenidos, nos sorprendio considerablemente como el contenido visual del video no fue tan relevante a la hora de la interpolacion, esto hizo que gran parte de los experimentos planteados sean redundantes, ya que los problemas que encontramos tienen todos la misma raiz, que es la naturaleza impredecible de la evolucion de un pixel a lo largo del tiempo. Esta evolucion no se ajusta completamente a ninguna de las curvas generadas por los metodos interpoladores analizados en el trabajo, efectivamente generando anomalias visuales.

Un ultimo punto que habria sido interesante analizar, habria sido tomar los cuadros generados por los metodos interpoladores, y aumentar la cantidad de cuadros por segundo proporcionalmente a los cuadros agregados. Esto nos habria dado un video con la misma "velocidad" que el original, pero con una fluidez mucho mas marcada, algunos televisores modernos cuentan con esta opcion y se usa principalmente para deportes.